\section{Conclusion and Future Work}
Based on the fact that users are only interested in part of a video scene at some cases, we designed a software approach named selective decoding to reduce the battery power consumption and increase the frame rate for HD video playback on mobile devices. Selective decoding is based on analyzing and tracing various intra- and inter-frame dependencies among macroblocks. By doing so, we compute a selective mask which indicates the macroblocks needed to present a clear scene in a user requested ROI and the modified decoder can then decode selectively according to the mask. 

Selective decoding presented illustrates the idea of achieving more efficient zoomable video playback by tracing the dependencies among macroblocks. There are many possible future work can be done. Firstly, the dependency file storage overhead is large. We store the dependency information as plain values in binary format. Better storage scheme and compression can be applied to reduce storage overhead. Secondly, the dependency file generation could be done online, which requires optimizing the process and integrating it with selective mask computation. Thirdly, expanding selective decoding to encoder may bring some benefits. The encoder can generate dependency files and control the amount of dependencies among macroblocks. Lastly, selective decoding at decoder may help to improve the existing research on zoomable in network streaming context, which deals with encoders mostly. 
