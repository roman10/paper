\section{Decoding Verification}
Selective decoding is a complicated process with lots of details despite its simple idea and principles. It is important to validate the selective decoding is done correctly. Several approaches can be used. 

The naive approach is simply watch the selective decoded video to see if there is any visible blemishes. A better approach is to compare the selectively decoded ROI pixel values with the pixels decoded by standard MPEG4 SP decoder. A third approach is to compare the output values of each stage of the selective decoding each the values produced by standard decoding. This approach does not only verify if selective decoding is done correctly, but also facilites finding the phase that causes problems if there is any. 



