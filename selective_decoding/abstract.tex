\begin{abstract}
Many mobile devices are not capable of playing high definition video smoothly. Moreover video playback on mobile devices drains the battery quickly. We propose a method named selective decoding to increase video playback frame rate and reduce power consumption. 
Selective decoding is based on the idea that users are usually interested in only part of a video scene. It will be more efficient if only the user's Region of Interest (ROI) is decoded. To enable selective decoding, we analyzed various dependency relationships among macroblocks in MPEG4 Part 2 Simple Profile codec, traced the macroblocks that are needed to decode the ROI, and modified standard decoding process to decode macroblocks selectively based on the trace. Our experiments on Android platform show that selective decoding can improve playback frame rate by up to 237.7\% and reduce energy consumption by up to 64.5\%.
\keywords{selective decoding, zoomable video, energy saving, Region of Interest playback}
\end{abstract}
